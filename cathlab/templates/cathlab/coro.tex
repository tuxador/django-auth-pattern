
\mainlanguage[fr]
\setuppagenumbering[alternative=singlesided,location={footer,inright},
                    right={ de \totalnumberofpages},
                    ]
\setuplayout[header=1cm,leftmargin=1.5cm,rightmargin=1.5cm,footer=1.5cm]
\definefontfeature[default][default][protrusion=quality,expansion=quality,script=latn]
\definefontfamily[kad][serif][Linux Libertine O]
\definefontfamily[kad][sans] [Gillius ADF]
\definefontfamily[kad][mono] [TeX Gyre Cursor][features=none]
\definefontfamily[kad][math] [TeX Gyre Pagella Math]


\definelayer[logo]	% name of the layer
    [x=0mm, y=0mm,	% from upper left corner of paper
      width=\paperwidth, height=\paperheight, % let the layer cover the full paper
      state=start,
    ]
\definelayer[sidelogo]	% name of the layer
    [x=0mm, y=0mm,	% from upper left corner of paper
     width=\paperwidth, height=\paperheight, % let the layer cover the full paper
     state=continue,
     ]

\setupbackgrounds[page][background={logo,sidelogo}]
%\setupbackgrounds[page][background={sidelogo}, state=continue]
    \setlayer[logo]
    [state=start,hoffset=6.5cm,voffset=1cm]{\externalfigure[kardia]}
 \setlayer[sidelogo]
    [state=continue,hoffset=1cm,voffset=1cm]{\externalfigure[kardia][width=2cm]}
\setupfootertexts[\setups{footertext}]

\startsetups footertext
  \framed[frame=off, align=normal, style={\ssx}]
      {Clinique {\sc Kardia}, Cité Sonelgaz  lot n°75/76 Gué de Constantine, Alger\\ Tél/Fax: 021 83 80 13/021 83 20 36 - Email: \from[kardia-email] }
\stopsetups

\setupbodyfont[kad,11pt]

\usesymbols[fontawesome]

\define\FA{\dosingleargument\doFA}
\def\doFA[#1]{\inlinedbox
    {\scale[height=1em]{\symbol[fontawesome][#1]}}} %to scale fontawesome with mainfont

\definecolor[darkgray]  [s=0.5]
\definecolor[lightgray] [s=0.95]
\definecolor[darkred]   [r=0.6]
\definecolor[darkgreen] [g=0.6]
\definecolor[darkblue]  [b=0.6]
\definecolor[cyan][r=0.02,g=0.49,b=0.52]

\definecolor[lightred]   [0.95(red,white)]
\definecolor[lightgreen] [0.95(green,white)]
\definecolor[lightblue]  [0.95(blue,white)]
\setuphead[subject][color=darkred,style=\ssa\bf] 
\setuphead[subsubject][color=darkred, style=\ss\bf]
\setuphead[subsubsubject][color=cyan, style=\ss]

\defineframedtext[leftbartext]
  [ width=broad,
    frame=off,
    framecolor=darkgray,
    leftframe=on,
    rulethickness=2ex,
    offset=0.25ex,
    loffset=3ex, 
    background=color,
    backgroundcolor=lightgray,
    ]
 \defineframedtext
  [exampletext]
  [leftbartext]
  [
    framecolor=cyan,
    backgroundcolor=lightgreen,
  ]

\defineframedtext
  [alerttext]
  [leftbartext]
  [
    framecolor=darkred,
    backgroundcolor=lightred,
  ]

\defineframedtext
  [blocktext]
  [leftbartext]
  [
    framecolor=darkblue,
    backgroundcolor=lightblue,
  ]   
\define\FA{\dosingleargument\doFA}
\def\doFA[#1]{\inlinedbox
    {\scale[height=1em]{\symbol[fontawesome][#1]}}}
\useURL[kardia-email][mailto:contact@cliniquekardia.com][][contact@cliniquekardia.com]
\definedescription
   [description]
   [headstyle=\ss\bf, headcolor=cyan, style=, location=hanging, width=broad, margin=2ex, alternative=hanging]
\startsetups table:width
  \setupTABLE[align={hyphenated,normal}]
  \setupTABLE[column][1][width=0.3\textwidth]
  \setupTABLE[column][2][width=0.4\textwidth]
  \setupTABLE[column][3][width=0.3\textwidth]
\stopsetups
% table des patients
\startsetups table:style
\setupTABLE[frame=off]
\setupTABLE[r][1][style=\ss]
\setupTABLE[r][2][style=\bf]
\setupTABLE[r][3][style=\ss]
\setupTABLE[c][3][align=left]
\setupTABLE[c][2][align=center]
\setupTABLE[r][4][style=\bf] % how to override the \tt in the second column?
\stopsetups
%protocole du stress
\startsetups table:protocole
\setupTABLE[frame=off,option=stretch]
      \setupTABLE[row][odd][background=color, backgroundcolor=lightgreen]
      \setupTABLE[row][1][style={\ss\bf}, background=color,
                             backgroundcolor=cyan, foregroundcolor=white]
      \setupTABLE[column][1][style={\ssx\bf}]
      \setupTABLE[column][5][align=right]
      \setupTABLE[column][3][align=center]

\stopsetups


\starttext


  \startalignment[middle]
 \dontleavehmode
\framed
  [frame=on,
    framecolor=darkred,
    background=color,
    backgroundcolor=,
    corner=round,
    offset=0.3cm,
   ]
  {\ssc \bf Coronarographie  Angioplastie  }
\stopalignment

\blank[1cm]

%\subject[patient]{Patient}

\startframed[frame=on,
    framecolor=darkred,
    background=color,
    backgroundcolor=,
    corner=round,
    offset=0.3cm,
   ]

%\placetable[none]{}
 {\bTABLE[setups={table:style, table:width}]
   \bTR
   \bTD nom \& prénoms \eTD
   \bTD date de naissance \eTD
   \bTD Adresse \eTD
   \eTR
   \bTR
   \bTD {{ patient.name }} \eTD
   \bTD {{ patient.birth }} \eTD
   \bTD {{ patient.adresse }} \eTD
   \eTR
   \bTR
   \bTD indication \eTD
   \bTD médecin correspondant\eTD
   \bTD assurance \eTD
   \eTR
\bTR
   \bTD {{ indication }} - \eTD
   \bTD {{ coronarographie.medecin }}\eTD
   \bTD CNAS CASNOS \eTD
   \eTR
\eTABLE} 
\stopframed
\startdescription{Abord artériel}
fémoraleautre radial 
\stopdescription
\startdescription{Matériel}
 {{ sonde }}  - {{ guide }}  - {{ stent }} 
\stopdescription

\startdescription{Produits}
 PDC {{ coronarographie.iode }}cc - Héparine {{ coronarographie.heparin }}cc - Loxen {{ coronarographie.loxen }}cc - Risordan {{ coronarographie.nitro }}cc.
\stopdescription
\subsubject{Réseau coronaire}
\startdescription{TCG}
{{ coronarographie.tcg }}
\stopdescription
\startdescription{IVA}
{{ coronarographie.iva }}
\stopdescription

\startdescription{Bissectrice}
{{ coronarographie.bissectrice }}
\stopdescription

\startdescription{Cx}
{{ coronarographie.circonflexe }}

\stopdescription
\startdescription{CD}

{{ coronarographie.coronaire_dte }} 
\stopdescription

%\bTABLE[setups={table:protocole}]
%
%\bTR \bTD \eTD \bTD Symptômes \eTD \bTD PA \eTD \bTD ECG \eTD \bTD echocoeur \eTD \eTR
%\bTR \bTD Base \eTD \bTD - \eTD \bTD 100/70 \eTD \bTD Normal \eTD \bTD  cinétique homogène\eTD\eTR
%\bTR \bTD Low-dose \eTD \bTD - \eTD \bTD 120/70 \eTD \bTD non modifié \eTD\bTD pas de modifications de la cinétique\eTD\eTR
%\bTR \bTD Peak-dose \eTD \bTD - \eTD \bTD 130/90 \eTD \bTD non modifié \eTD \bTD Cinétique normale\eTD \eTR
%\bTR \bTD Récup. \eTD \bTD - \eTD \bTD 100/70 \eTD \bTD non modifié \eTD \bTD cinétique homogène \eTD\eTR
%\eTABLE

\subsubject[conclusion]{Conclusion}
\startalerttext
{{ coronarographie.conclusion }}
\stopalerttext
\subsubsubject[décision thérapeutique]{Décision thérapeutique}
\startexampletext
  {{ coronarographie.decision }} .
\stopexampletext

\subsubject{Procédure}
{{ coronarographie.procedure }}
\subsubject[dispositios]{Dispositions complémentaires}
\startalerttext
{{ coronarographie.dispositions }}
\stopalerttext

\blank[1cm]
\raggedleft{Dr {\sc Yahyaoui M K} --  Cardiologue\\
  Alger le \currentdate \hspace[2em]
  
\stoptext
   
