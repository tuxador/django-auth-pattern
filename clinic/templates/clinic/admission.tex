

\setuppagenumbering[alternative=singlesided,location={footer,inright}]
\setuplayout[header=1cm,leftmargin=1.5cm,rightmargin=1.5cm,footer=1.5cm]
\mainlanguage[fr]
\definefontfeature[default][default][protrusion=quality,expansion=quality,script=latn]
\definefontfamily[kad][serif][Minion Pro]
\definefontfamily[kad][sans] [Gillius ADF]
\definefontfamily[kad][mono] [TeX Gyre Cursor][features=none]
\definefontfamily[kad][math] [TeX Gyre Pagella Math]


\definelayer[logo]	% name of the layer
    [x=0mm, y=0mm,	% from upper left corner of paper
      width=\paperwidth, height=\paperheight, % let the layer cover the full paper
      state=start,
    ]
\definelayer[sidelogo]	% name of the layer
    [x=0mm, y=0mm,	% from upper left corner of paper
     width=\paperwidth, height=\paperheight, % let the layer cover the full paper
     state=continue,
     ]
\setupexternalfigures[directory={ , /home/kaddour/Documents}]
\setupbackgrounds[page][background={logo,sidelogo}]
%\setupbackgrounds[page][background={sidelogo}, state=continue]
    \setlayer[logo]
    [state=start,hoffset=6.5cm,voffset=1cm]{\externalfigure[kardia.jpg]}
 \setlayer[sidelogo]
    [state=continue,hoffset=1cm,voffset=1cm]{\externalfigure[kardia.jpg][width=4cm]}
\setupfootertexts[\setups{footertext}]

\startsetups footertext
  \framed[frame=off, align=normal, style={\ssx}]
      {Clinique {\sc Kardia}, Cité Sonelgaz  lot n°75/76 Gué de Constantine ,Alger\\ Tél/Fax: 021 83 80 13/021 83 20 36 - Email: contact(at)cliniquekardia.com}
\stopsetups



\setupbodyfont[kad,11pt]

\usesymbols[fontawesome]
\definecolor[darkgray]  [s=0.5]
\definecolor[lightgray] [s=0.95]
\definecolor[darkred]   [r=0.5]
\definecolor[darkgreen] [g=0.5]
\definecolor[darkblue]  [b=0.5]
\definecolor[cyan][r=0.02,g=0.49,b=0.52]

\definecolor[lightred]   [0.95(red,white)]
\definecolor[lightgreen] [0.95(green,white)]
\definecolor[lightblue]  [0.95(blue,white)]
\setuphead[subject][color=darkred,style=\ssa\bf] 
\setuphead[subsubject][color=darkred, style=\bf]
\setuphead[subsubsubject][color=cyan, style=\ss]

\defineframedtext[leftbartext]
  [ width=broad,
    frame=off,
    framecolor=darkgray,
    leftframe=on,
    rulethickness=2ex,
    offset=0.25ex,
    loffset=3ex, 
    background=color,
    backgroundcolor=lightgray,
    ]
 \defineframedtext
  [exampletext]
  [leftbartext]
  [
    framecolor=cyan,
    backgroundcolor=lightgreen,
  ]

\defineframedtext
  [alerttext]
  [leftbartext]
  [
    framecolor=darkred,
    backgroundcolor=lightred,
  ]

\defineframedtext
  [blocktext]
  [leftbartext]
  [
    framecolor=darkblue,
    backgroundcolor=lightblue,
  ]   
\define\FA{\dosingleargument\doFA}
\def\doFA[#1]{\inlinedbox
    {\scale[height=1em]{\symbol[fontawesome][#1]}}}

\definedescription
   [description]
   [headstyle=\ssx\bf, style=, location=hanging, width=broad, margin=2ex, alternative=hanging]
\startsetups table:width
  \setupTABLE[align={hyphenated,normal}]
  \setupTABLE[column][1][width=0.3\textwidth]
  \setupTABLE[column][2][width=0.4\textwidth]
  \setupTABLE[column][3][width=0.3\textwidth]
\stopsetups
\startsetups table:style
\setupTABLE[frame=off]
\setupTABLE[r][1][style=\ss]
\setupTABLE[r][2][style=\tfa\bf]
\setupTABLE[r][3][style=\ss]
\setupTABLE[c][3][align=left]
\setupTABLE[c][2][align=center]
\setupTABLE[r][4][style=\bf] % how to override the \tt in the second column?
\stopsetups

\starttext

  \startalignment[middle]
 \dontleavehmode
\framed
  [frame=on,
    framecolor=darkred,
    background=color,
    backgroundcolor=,
    corner=round,
    offset=0.3cm,
   ]
  {\ssc \bf RAPPORT MEDICAL}
\stopalignment

\blank[2cm]

%\subject[patient]{Patient}

\startframed[frame=on,
    framecolor=darkred,
    background=color,
    backgroundcolor=,
    corner=round,
    offset=0.3cm,
   ]

%\placetable[none]{}
 {\bTABLE[setups={table:style, table:width}]
   \bTR
   \bTD nom \& prénoms \eTD
   \bTD date de naissance \eTD
   \bTD Adresse \eTD
   \eTR
   \bTR
   \bTD {{ patient.name }} \eTD
   \bTD {{ patient.birth }} \eTD
   \bTD {{ patient.adresse}} \eTD
   \eTR
   \bTR
   \bTD motif d'admission \eTD
   \bTD médecin correspondant\eTD
   \bTD assurance \eTD
   \eTR
\bTR
   \bTD {{ motif }} - 
   \eTD
   \bTD {{ admission.med_ref }}-- 
   \eTD
   \bTD CNAS
        CASNOS
        autre
        non assuré
        
\eTD
   \eTR
\eTABLE} 
\stopframed
\blank[1cm]
MadameMonsieur {{ patient.name }} née le {{ patient.birth }} a séjourné à notre niveau du {{ admission.admission_date }} au {{ admission.sortie }}\currentdate pour {{ motif }} - .
\subsubject[antécedents]{Antécedents}
\subsubsubject[facteurs-de-risque-cv]{facteurs de risque CV}
\startitemize[2,columns,two,packed]
\startitem  Hypertension artérielle. \stopitem 
 \startitem  Diabète \stopitem
 \startitem  Dyslipidémie \stopitem
 \startitem  tabagique \stopitem
 \startitem  Obésité \stopitem
 \startitem  Sédentarité \stopitem
 \startitem  Hérédité coronarienne \stopitem
\stopitemize

\subsubsubject[médicaux]{médicaux}
{{ patient.history }}pas d'antécédents particuliers .
\subsubject[histoire-maladie]{Histoire de la maladie}
{{ admission.histoire }}
\subsubject[examen-clinique]{Examen clinique}

\startitemize[2,columns,two,packed]
  \startitem
   eupnéique{\bf dyspnée stade {{ admission.dyspnea_nyha }} de la NYHA}
          \stopitem
  \startitem fièvre.\stopitem
  \startitem   Absence d'angor.Angor au stade {{ admission.angina_scc }} de la {\it SCC}.
  \stopitem
   \startitem Syncope.\stopitem
\stopitemize
\startdescription{Auscultation}
  L'examen clinique retrouve à l'auscultation: {{ admission.auscultation }} avec une fréquence cardiaque à = {{ admission.heart_rate}} Bpm.
\stopdescription
\startdescription{En périphérie}
  les pouls sont {{ admission.pulse }} avec une PA = {{ admission.systolic_bp }}/{{ admission.diastolic_bp }}mmHg.
  Il existe des signes d'insuffisance cardiaque gauche.Absence de signes d'insuffisance cardiaque gauche
  On note aussi des signes d'insuffisance cardiaque droite turgescence spontanée des jugulaires reflux hépato jugulaire oedèmes des membres inférieurs .Absence de signes d'insuffisance cardiaque droite.
\stopdescription
\subsubsubject[ECG]{ECG}
Le rythme est en sinusalfibrillation atrialeflutter auriculairetacchycardie ventriculaireélectro stimulé
          à fréquence cardiaque de {{ admission.freq }} puls/min.

          Il existe un bloc de branche gauche.
          Il existe un bloc de branche droit.
          
          On note aussi une hypertrophie ventriculaire gauche électrique.On note aussi une hypertrophie ventriculaire droite électrique.
          
      Sur le plan ischémique:  une lésion en septal une lésion en antérieur une lésion en latéral une lésion en postérieurabsence de lésions électriques,
               une ischémie en septal une ischémie en antérieur une ischémie en latéral une ischémie postérieurabsence d'ischémie électrique,
                       onde T inversée en septal onde T inversée en antérieur onde T inversée en en latéral onde T inversée en postérieur. \\
            Le QT corrigé est estimé à {{ admission.corrected_qt }} msec. \\
            
            
              \subsubject[téléthorax]{Téléthorax}
              Le téléthorax retrouve: {{ admission.telethorax }} \\
              
              

  
\subsubject[echo-coeur]{Echo coeur}
Faite le {{ admission.date_echo }}: {{ admission.echocoeur }} 
 {\bf l'ETO} retrouve: {{admission.eto }}
\subsubject[biologie]{Biologie}
voir bilans

\hairline
\subsubject[total]{Au total}
\startalerttext
 {{ admission.resume }}
\stopalerttext
\subsubsubject[evolution]{Evolution}
{{ admission.evolution }}
\subsubsubject[traitement]{Traitement de sortie}
{{ admission.ordonnance }}
\subsubsubject[dispositions-complémentaires]{dispositions complémentaires}
\startexampletext
{{ admission.dispositions }}
\stopexampletext
\blank[2cm]
\raggedleft{Dr {\sc Yahyaoui M K} --  Cardiologue\\
  Alger le \currentdate \hspace[2em]
  }
\stoptext
                
